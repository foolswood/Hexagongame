\documentclass[10pt,twocolumn]{article}
\usepackage{hyperref}
\usepackage{graphicx}
\usepackage{color}
\usepackage[a4paper,margin=20mm]{geometry}
\begin{document}
\title{How many colours does Chromagon need?}
\author{David Honour}
\maketitle

\section{What is Chromagon?}
A maze consists of some coloured regions separated by coloured dividers.
A coloured player marker starts in one of the regions.
The player marker may move to an adjacent region cross any line whose colour
matches its own.
When the marker moves it takes on the colour of the region that it was in at
the start of the move.

See \url{http://foolswood.github.io/Hexagongame/} for an interactive version of
these rules applied to tiled hexagons.
\subsection{What is a necessary colour?}
Mazes are considered equivalent if all their spatial paths are the same.
A trivial example of equivalent mazes would be if all the regions and lines
(elements) of one colour were to swap colour with all the elements of another.
\subsection{1D (straight line) mazes}
The simplest maze type that the rules can be applied to is one where each
region borders 2 others. These are equivalent to a single row maze of any
other shape.

\section{Proving a maze cannot be reduced}
\label{irr3}
\begin{figure}
\caption{A 1D maze requiring 3 colours}
\label{requires3}
\centering
\input{irreducible3.eps_tex}
\end{figure}
Consider the maze shown in figure \ref{requires3} where $S$ is the starting
colour, $A$, $B$ and $C$ are the colours of the corresponding regions and $x$
and $y$ are the colours of the dividers.

This maze can be expressed in terms of the set of equalities and inequalities
that are required to impose the required movement restrictions:
\begin{equation}S = x\end{equation}
\begin{equation}A \neq x \label{3neq}\end{equation}
\begin{equation}A = y\end{equation}
\begin{equation}B = y \label{3eq}\end{equation}
\begin{equation}C \neq y \label{3cny}\end{equation}
\begin{equation}C \neq x \label{3cnx}\end{equation}

Substituting equation \ref{3eq} into \ref{3neq} yields:
\begin{equation}y \neq x \label{3ynx}\end{equation}
which when considered with equations \ref{3cny} and \ref{3cnx} can be seen to
require at least 3 colours to solve.

\section{The maximum number of colours required for any maze}
The number of colours required by a given maze is determined by the set of
inequalities that restrict the available paths.
Any equalities will force colour reuse (and therefore cannot increase the total
number of required colours).

For instance to require 2 colours requires $a \neq b$, but to require 3 (as in
section \ref{irr3}) requires a set of inequalities of the form:
\begin{equation}
a \neq b
\quad \textrm{and} \quad
b \neq c
\quad \textrm{and} \quad
c \neq a
\end{equation}
Note that in order to have a set of equations like this it is necessary to have
2 things that are simultaneously $\neq a$.

More generally, in order to require a $n$ colours a single element must be able
to participate simultaneously in $n-1$ inequalities.

\subsection{1D}
\begin{figure}
\caption{A section of a 1D maze showing every region that can interact with $y$}
\centering
\input{labelled4.eps_tex}
\label{1dall}
\end{figure}
Consider the interactions of $y$ in figure \ref{1dall}.

For $C$ and $D$ to have any effect at $y$ the path must turn around at $D$.
In order for this to be possible:
\begin{equation} B = C \end{equation}
which implies that the possible inequalities $B \neq y$ and $C \neq y$ are
redundant.

If $A \neq y$ and $S \neq y$ then $y$ cannot be traversed and thus no
inequalities referring to $B$, $C$ or $D$ can take effect.

If $S \neq y$ then to traverse $y$ and bring the remaining potential
inequalities ($B \neq y$ and $D \neq y$) into play requires the path to return
to $B$ which requires $x = B = S$ making $B \neq y$ and $S \neq y$ redundant.

In order to return to the line from $A$, and thus apply $A \neq y$ requires
that $x = B = S$, again this makes the constraint redundant.

Therefore the maximum number of simultaneous inequalities that can be affect
$y$ in any maze is 2, and thus 3 colours is the maximum required to express
any 1D maze.
\end{document}
